\documentclass[12pt]{article} % article, report, or book
\usepackage{geometry}                		% See geometry.pdf to learn the layout options. There are lots.
\geometry{letterpaper}                   		% ... or a4paper or a5paper or ... 
%\usepackage[parfill]{parskip}    		% Activate to begin paragraphs with an empty line rather than an indent
\usepackage{graphicx}				% Use pdf, png, jpg, or eps§ with pdflatex; use eps in DVI mode
								% TeX will automatically convert eps --> pdf in pdflatex		
\usepackage{amsmath}
%\graphicspath{{/Users/lukebury/Documents/School/CU/ORCCA/}}
\usepackage{color}
\graphicspath{{../figures/}}

\usepackage[utf8]{inputenc} % allows for é
\usepackage{float} % allows for [H]

\newcommand{\paragraphtitle}[1]{\paragraph{#1}\mbox{}\\}
%=======================================================================================================
\title{APPM 5460 Final Project Proposal}
\author{Luke Bury \& Don Kuettel}
%=======================================================================================================
\begin{document}
\maketitle
%=======================================================================================================
\section*{Introduction}
This proposal aims to investigate homoclinic orbits in the dynamical system known as the Circular Restricted Three-Body Problem (CR3PB). The CR3PB, further described in the following section, is a classical astrodynamics problem that has been studied for over 200 years and contains a plethora of interesting dynamical phenomena, including homoclinic orbits. 

In mathematics, a homoclinic orbit is defined as a trajectory of a flow of a dynamical system which joins a saddle equilibrium point to itself. More precisely, a homoclinic orbit lies in the intersection of the stable manifold, $W^s(p)$, and the unstable manifold, $W^u(p)$, of an equilibrium. Figure \ref{f:homoclinic_example} shows an example of a simple, two-dimensional homoclinic orbit about the saddle equilibrium point $p$. As the figure shows, as time approaches either negative or positive infinity, the homoclinic orbit will approach $p$. 

\begin{figure}[H]
    \centering
    \includegraphics[width=4in]{homoclinic_orbit.png}
    \caption{This figure depicts a 2D homoclinic orbit.}
    \label{f:homoclinic_example}
\end{figure}

Homoclinic orbits, first discovered by Henri Poincar\'{e} in the 1885 Acta Mathematica competition sponsored by King Oscar II of Sweden, play an important role in the chaotic behavior of a dynamical system. Lying on the intersection between a stable and unstable manifold of the same equilibrium point, or orbit, the geometry of homoclinic orbits (i.e., the geometry of the manifold intersection) offers a way in which simple local information can be extrapolated to complicated global behavior. This proposal looks to study Poincar\'{e}'s work on homoclinic orbits and use that knowledge to find examples of homoclinic orbits in the CR3BP.


%-----------------------------------------------------------------------------------------
\section*{History of the Problem} % History / Lit Review
As documented by Andersson and Barrow-Green \cite{Andersson1994,BarrowGreen1994}, in 1885, Acta Mathematica announced a mathematics competition to the world. This competition, sponsored by King Oscar II of Sweden, encouraged interested parties to make an attempt at solving one of four selected problems. Henri Poincaré, a prominant mathematician of the time (who was largely favored to win the competition) attempted the first problem, which essentially asked for a solution to the troubling n-body problem. However, Poincaré decided to first attempt the three-body problem, since it was the first order of the problem remaining unsolved. To further simplify his initial effort, he restricted the three-body system in a manner known today as the Circular Restricted Three Body Problem (CR3BP), shown in Figure \ref{fig:CR3BP}. In the CR3BP, the origin is set at the barycenter of two bodies of interest (eg, Earth \& Moon), and the frame rotates so these bodies remain stationary on the x-axis. The bodies are assumed to move in perfectly circular orbits, and act as point masses from a gravitational perspective. The system is typically normalized so that the masses of the two bodies sum to 1 ($m_1 = \mu,\; m_2 = 1-\mu$), the distance between the bodies is 1, and the gravitational constant G is equal to 1. Under these conditions, the equations of motion for the CR3BP are shown in equations (\ref{eomx}-\ref{eomz})
\begin{align}
\ddot{x} &=  2\dot{y} + x - (1-\mu)\left(\dfrac{x+\mu}{R_1^3}\right) - \mu\left(\dfrac{x-1+\mu}{R_2^3}\right) \label{eomx}\\
\ddot{y} &= - 2\dot{x} + y\left(-\dfrac{1-\mu}{R_1^3} - \dfrac{\mu}{R_2^3} + 1\right) \label{eomy}\\
\ddot{z} &= z\left(-\dfrac{1 - \mu}{R_1^3} - \dfrac{\mu}{R_2^3}\right) \label{eomz}
\end{align}
Poincaré submitted his work on the CR3BP, won the competition, and collected the 2500 Crown prize. However, around the time that his work was first being printed, a discussion with Lars Edvard Phragmén led to the discovery of an error within Poincaré's submission that held significant ramifications. The error was rooted in Poincaré's failing "to take proper account of the exact geometric nature of a particular curve" \cite{BarrowGreen1997}. In Theorem III of the paper's first, and flawed, edition, Poincaré claimed that a particular invariant curve was closed (Figure \ref{fig:curveIntersection1}(a,b)). He failed to consider that the curve could be self-intersecting (Figure \ref{fig:curveIntersection1}(c)) - a mistake whose discovery cost him the fortune he had won, but also cemented his place among legendary mathematicians for his resulting discovery of \textit{doubly asymptotic}, or \textit{homoclinic}, points in dynamical systems.
\begin{figure}[H]
\centering
\includegraphics[scale=0.4]{CR3BP.png}\nonumber
\caption{Layout of Circular Restricted Three Body Problem \cite{KoonLoMarsdenRoss2011}}
\label{fig:CR3BP}
\end{figure}
\begin{figure}[H]
\centering
\includegraphics[scale=0.4]{curveIntersection1.png}\nonumber
\caption{(a) Diagrams of incorrectly closed invariant curves from the first edition of Poincaré's memoir. (b)Invariant curve with self intersection from Poincaré's corrected work \cite{BarrowGreen1997}}
\label{fig:curveIntersection1}
\end{figure}

%-----------------------------------------------------------------------------------------
\section*{Application of Poincaré's Results and Project Goals} 
Based on these famous results of Poincaré's work, our project will involve locating the intersection points of stable and unstable manifolds for our system of differential equations in order to find homoclinic trajectories. Our first goal will be to use this method to recreate the trajectory shown in Figure \ref{fig:homoclinicPoints}. The second goal of our project will be to investigate how these manifolds and homoclinic trajectories change with small deviations in $\mu$, the mass ratio for a 3-body system.
\begin{figure}[H]
\centering
\includegraphics[scale=0.4]{homoclinicPoints.png}\nonumber
\caption{(a) A group of four homoclinic points lying at the intersection of stable (green) and unstable (red) manifolds for the Sun-Jupiter system shown in (b) \cite{KoonLoMarsdenRoss2011}}
\label{fig:homoclinicPoints}
\end{figure}


%=======================================================================================================
\newpage
\bibliographystyle{plain}
\bibliography{../bibliography/appm5460.bib}

\end{document}  