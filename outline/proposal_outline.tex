\documentclass{article}
\usepackage{geometry}                		% See geometry.pdf to learn the layout options. There are lots.
\geometry{letterpaper}                   		% ... or a4paper or a5paper or ... 
%\usepackage[parfill]{parskip}    		% Activate to begin paragraphs with an empty line rather than an indent
\usepackage{graphicx}				% Use pdf, png, jpg, or eps§ with pdflatex; use eps in DVI mode
								% TeX will automatically convert eps --> pdf in pdflatex		
\usepackage{amsmath}
%\graphicspath{{/Users/lukebury/Documents/School/CU/ORCCA/}}
\usepackage{color}

\newcommand{\paragraphtitle}[1]{\paragraph{#1}\mbox{}\\}

\title{Outline - APPM 5460 Proposal}
\author{Luke Bury \& Don Kuettel}


\begin{document}
\maketitle


%=======================================================================================================
\subsection{needs}
\begin{itemize}
  \item 2-4 pages
  \item Lit review
  \item dynamical system modeled by ordinary differential equations
  \item material from class\\\\\\
\end{itemize}

\subsection{outline}
\begin{itemize}
	\item Intro
	  \begin{itemize}
	  	\item In this proposal, we will be investigating homoclinic orbits in the Circular Restricted Three-Body Problem
	  	\item (tie to competition history)
	  	\item Homoclinic orbits are ... 
	  	\item They are important because ...
	  \end{itemize}
	\item History / Lit Review
	  \begin{itemize}
	  	\item Competition basics ... 4 problems
	  	\item 1st problem was n-body problem
	  	\item Poincare went for 3-body since it was the first unsolved... settled for CR3BP
	  	\item (description of CR3BP)
	  	\item \color{red}$\ddot{x} =  2\dot{y} + x + \left(\dfrac{1-\mu}{r_1^3}-\dfrac{3R_1^2 J_{2,1}(1-\mu)}{2 r_1^7}(5z^2 - r_1^2)\right)(x_1- x) + \left(\dfrac{\mu}{r_2^3}-\dfrac{3 \mu R_2^2 J_{2,2}}{2 r_2^7}(5z^2 - r_2^2)\right)(x_2 - x)\nonumber\\
		\ddot{y} = - 2\dot{x} + y\left(-\dfrac{1-\mu}{r_1^3} - \dfrac{\mu}{r_2^3} + \dfrac{3R_1^2 J_{2,1}(1-\mu)}{2 r_1^7}(5z^2 - r_1^2) + \dfrac{3 \mu R_2^2 J_{2,2}}{2 r_2^7}(5z^2 - r_2^2) + 1\right) \\
		\ddot{z} = z\left(-\dfrac{1 - \mu}{r_1^3} - \dfrac{\mu}{r_2^3} + \dfrac{3R_1^2 J_{2,1}(1-\mu)}{2 r_1^7}(5z^2-3r_1^2) + \dfrac{3\mu R_2^2 J_{2,2}}{2 r_2^7}(5z^2-3r_2^2)\right)\nonumber$\color{black}
	  	\item (a bit on the error ... discuss the math)
	  	\item as a result, found homoclinic points/orbits

	  \end{itemize}
	\item Application
	  \begin{itemize}
	  	\item \color{red}Much research has been conducted in this field since (references, references, references)\color{black}
	  	\item (pg 72+ of book)
	  	\item 
	  \end{itemize}
\end{itemize}


\end{document}  